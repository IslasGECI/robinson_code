\documentclass{article} % El documento es de tipo artículo
\usepackage[utf8]{inputenc} % Paquete que permite escribir caracteres especiales
\usepackage[english, spanish, es-tabla]{babel} % Paquete para cambiar "Cuadro" a "Tabla" en encabezados de tablas
\usepackage{graphicx} % Paquete para importar figuras
\usepackage[hidelinks]{hyperref} % Paquete para agregar vínculos como enlaces
\usepackage{booktabs}
\usepackage{authblk}
\usepackage{amsmath}
\usepackage{float}
\usepackage{xcolor}
\usepackage{caption}
\usepackage{csvsimple}
\usepackage{pythontex}
\usepackage{pgfplotstable}
\usepackage{adjustbox}
\usepackage[a4paper,top=2cm,bottom=2cm,left=3cm,right=3cm,marginparwidth=1.75cm]{geometry}
\usepackage{parcolumns}

\author{Dirección de Ciencia de Datos}

\title{Estimación de tamaño poblacional de gato feral en Isla Robinson Crusoe\\ \begin{large} Grupo de Ecología y Conservación de Islas, A.C. \end{large}}

\begin{document}

\maketitle

\pgfplotstableread[col sep=comma]{../preds_1km_grid-cats.csv}\Cats

\begin{table}[H]
    \centering
    \caption{Tamaño poblacional gato feral en Isla Robinson Crusoe.
    La primera sesión corresponde al mes de octubre del 2021.}
     \pgfplotstabletypeset[
       string type,
       columns={.season,N,lcl,ucl},
       columns/{N}/.style={column name={\textbf{N}}},
       columns/{.season}/.style={column name={\textbf{Sesión}}},
       columns/{lcl}/.style={column name={\textbf{Límite inferior}}},
       columns/{ucl}/.style={column name={\textbf{Límite superior}}},
      every head row/.style={before row=\toprule,after row=\midrule},
      every last row/.style={after row=\bottomrule},
      ]{\Cats}
    \label{tab:csvEstimacionGatos}
  \end{table}

\end{document}
