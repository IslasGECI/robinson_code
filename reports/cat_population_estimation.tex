\documentclass{article} % El documento es de tipo artículo
\usepackage[utf8]{inputenc} % Paquete que permite escribir caracteres especiales
\usepackage[english, spanish, es-tabla]{babel} % Paquete para cambiar "Cuadro" a "Tabla" en encabezados de tablas
\usepackage{graphicx} % Paquete para importar figuras
\usepackage[hidelinks]{hyperref} % Paquete para agregar vínculos como enlaces
\usepackage{booktabs}
\usepackage{authblk}
\usepackage{amsmath}
\usepackage{float}
\usepackage{xcolor}
\usepackage{caption}
\usepackage{csvsimple}
\usepackage{pgfplotstable}
\usepackage{adjustbox}
\usepackage[a4paper,top=2cm,bottom=2cm,left=3cm,right=3cm,marginparwidth=1.75cm]{geometry}
\usepackage{parcolumns}

\pgfplotstableread[col sep=comma]{../prediction_with_count_cells.csv}\Cats

\author{Dirección de Ciencia de Datos}

\title{Estimación de tamaño poblacional de gato feral en Isla Robinson Crusoe hasta julio de 2022\\ \begin{large} Grupo de Ecología y Conservación de Islas, A.C. \end{large}}

\begin{document}

\maketitle

\begin{abstract}
  Hubo hasta \textbf{66 individuos} de gato feral en Isla Robinson Crusoe en julio de 2022.
  Las estimaciones de octubre de 2021 a julio de 2022 oscilaron entre 12 y 125 individuos.
\end{abstract}

\section*{Metodología}

\subsection*{Datos}
Calculamos las predicciones a partir de los datos obtenidos por las cámaras trampa.
Esos datos incluyen registros desde octubre de 2021 hasta julio de 2022.
Los resultados actuales corresponden a los datos recibidos el 26 de octubre de 2022.

\subsection*{Modelo}
Usamos el paquete \href{https://github.com/eradicate-dev/eradicate}{\texttt{eradicate}} desarrollado por David Ramsey\footnote{\href{https://www.ari.vic.gov.au/about-us/staff}{Arthur Rylah Institute}}. 
El paquete \texttt{eradicate} implementa modelos mixtos para estimar el tamaño de la población a partir de conteos espaciales
 ({\href{https://onlinelibrary.wiley.com/doi/10.1111/j.0006-341X.2004.00142.x}{Royle, 2004}}).
El diámetro del rango hogareño de gato feral utilizado por el modelo es de 1 km.

\section*{Resultados}

\begin{table}[H]
    \centering
    \caption{Tamaño poblacional de gato feral en Isla Robinson Crusoe.
    La sesión 1 corresponde al mes de octubre del 2021, 
    la sesión 4 a enero de 2022 y la sesión 10 a julio de 2022.
    \textbf{N} se refiere al tamaño de la población.}
     \pgfplotstabletypeset[
       string type,
       columns={.season,N,lcl,ucl,cells_with_camera_data,percentage_of_cells_with_camera_data},
       columns/{N}/.style={column name={\textbf{N}}},
       columns/{.season}/.style={column name={\textbf{Sesión}}},
       columns/{lcl}/.style={column name={\textbf{Límite inferior}}},
       columns/{ucl}/.style={column name={\textbf{Límite superior}}},
       columns/{cells_with_camera_data}/.style={column name={\textbf{Celdas con datos}}},
       columns/{percentage_of_cells_with_camera_data}/.style={column name={\textbf{\% Celdas con datos}}},
      every head row/.style={before row=\toprule,after row=\midrule},
      every last row/.style={after row=\bottomrule},
      ]{\Cats}
    \label{tab:csvEstimacionGatos}
  \end{table}

\end{document}
