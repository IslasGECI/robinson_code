\documentclass{article} % El documento es de tipo artículo
\usepackage[utf8]{inputenc} % Paquete que permite escribir caracteres especiales
\usepackage[english, spanish, es-tabla]{babel} % Paquete para cambiar "Cuadro" a "Tabla" en encabezados de tablas
\usepackage{graphicx} % Paquete para importar figuras
\usepackage[hidelinks]{hyperref} % Paquete para agregar vínculos como enlaces
\usepackage{booktabs}
\usepackage{authblk}
\usepackage{amsmath}
\usepackage{float}
\usepackage{xcolor}
\usepackage{caption}
\usepackage{csvsimple}
\usepackage{pgfplotstable}
\usepackage{adjustbox}
\usepackage[a4paper,top=2cm,bottom=2cm,left=3cm,right=3cm,marginparwidth=1.75cm]{geometry}
\usepackage{parcolumns}
\usepackage[lang=english]{bicaption}

\pgfplotstableread[col sep=comma]{../prediction_with_count_cells_coatis.csv}\Coatis
\pgfplotstableread[col sep=comma]{../data/count_of_sessions_per_grid.csv}\CamerasSession

\author{ Grupo de Ecología y Conservación de Islas, A.C. }

\title{Distribución y abundancia de coati en Isla Robinson Crusoe hasta \VAR{fecha_prediccion}}

\begin{document}

\maketitle

\begin{abstract}
  Hubo hasta \textbf{\VAR{prediction} individuos de coati} en Isla Robinson Crusoe en \VAR{fecha_prediccion}.
  Las estimaciones de \VAR{fecha_inicio} a \VAR{fecha_prediccion} oscilaron entre \VAR{min} y \VAR{max} individuos.

\end{abstract}
\selectlanguage{english}
\begin{abstract}
  Up to \textbf{\VAR{prediction} coati individuals} were present on Robinson Island in \VAR{prediction_date}.
  Estimates from \VAR{start_date} to \VAR{prediction_date} ranged from \VAR{min} to \VAR{max} individuals.
\end{abstract}
\selectlanguage{spanish}

\begin{parcolumns}{2}
  \colchunk[1]{\section*{Metodología}
  \subsection*{Datos}
Calculamos la distribución y abundacia de la población de coati a partir de los datos de las cámaras trampa.
Esos datos incluyen registros desde \VAR{fecha_inicio} hasta \VAR{fecha_fin}.
Los resultados actuales incluyen los datos recibidos el \VAR{fecha_recepcion_datos}.
}
  \colchunk[2]{\section*{Methodology}
  \subsection*{Data}
We calculated the distribution and abundance of the coati population using the camera-trap data. 
This data includes records from \VAR{start_date} through \VAR{end_date}.
The current results include the data set received on \VAR{data_reception_date}.
}
\end{parcolumns}

\begin{parcolumns}{2}
\colchunk[1]{\subsection*{Modelo}
Usamos el paquete \href{https://github.com/eradicate-dev/eradicate}{\texttt{eradicate}} \cite{ram2022} desarrollado por David Ramsey\footnotemark.
El paquete implementa modelos mixtos para estimar el tamaño de la población a partir de conteos espaciales
 {\href{https://onlinelibrary.wiley.com/doi/10.1111/j.0006-341X.2004.00142.x}{\cite{roy2004}}}.
El diámetro del rango hogareño de coati utilizado por el modelo es de 1 km.
  }\footnotetext{\href{https://www.ari.vic.gov.au/about-us/staff}{Arthur Rylah Institute}}
  \colchunk[2]{\subsection*{Model}
We used the package \href{https://github.com/eradicate-dev/eradicate}{\texttt{eradicate}}
\cite{ram2022} developed by David Ramsey\footnotemark[1].
The package implements N-mixture models to estimate population size {\href{https://onlinelibrary.wiley.com/doi/10.1111/j.0006-341X.2004.00142.x}{\cite{roy2004}}}.
We used a home range diameter of 1 km for the coati population.
}
\end{parcolumns}


\begin{parcolumns}{2}
  \colchunk[1]{\section*{Resultados}
  La Tabla \ref{tab:csvEstimacionCoati} muestra el tamaño de la
  población para cada mes desde octubre de
  2021 hasta \VAR{fecha_fin}.
  La mediana de \VAR{fecha_inicio} a \VAR{fecha_fin} es
  de \textbf{\VAR{median} individuos}. Durante todo el periodo las
  estimaciones oscilaron \textbf{entre \VAR{min} y \VAR{max}
  individuos}. El límite superior para \VAR{fecha_prediccion}
  es de \textbf{\VAR{prediction} individuos}.
  La Figura \ref{fig:coati_population_distribution} muestra la distribución de la
  población de coati usando los datos de \VAR{fecha_inicio} a \VAR{fecha_fin}.
  La Figura \ref{fig:cameras_per_month} muestra la localización de las 
  cámaras trampa con datos en \VAR{fecha_prediccion}.
}
  \colchunk[2]{\section*{Results}
  Table \ref{tab:csvEstimacionCoati} shows the size of the population for
  each month from \VAR{start_date} to \VAR{end_date}.
  The median for the period of \VAR{start_date}
  to \VAR{prediction_date} was \textbf{\VAR{median} individuals}. Throughout
  the period, the estimates ranged \textbf{from \VAR{min} to
  \VAR{max} individuals}. The upper limit for \VAR{prediction_date}
  was \textbf{\VAR{prediction} individuals}.
  Figure \ref{fig:coati_population_distribution} shows the distribution of the
  coati population using data of the period from \VAR{start_date} to \VAR{end_date}.
  Figure \ref{fig:cameras_per_month} shows the
  locations of the camera traps with data for \VAR{prediction_date}
}
\end{parcolumns}

\begin{table}[H]
  \centering
  \bicaption{Tamaño poblacional de coati en Isla Robinson Crusoe.
  \texttt{N} es el número de individuos de coati.}{
    Coati population size on Robinson Crusoe Island.
    \texttt{N} is the number of coati individuals.
  }
   \pgfplotstabletypeset[
     string type,
     columns={months,N,lcl,ucl,cells_with_camera_data,percentage_of_cells_with_camera_data},
     columns/{months}/.style={column name={\textbf{Mes}}},
     columns/{N}/.style={column name={\textbf{N}}},
     columns/{lcl}/.style={column name={\textbf{Límite inferior}}},
     columns/{ucl}/.style={column name={\textbf{Límite superior}}},
     columns/{cells_with_camera_data}/.style={column name={\textbf{Celdas con datos}}},
     columns/{percentage_of_cells_with_camera_data}/.style={column name={\textbf{\% Celdas con datos}}},
    every head row/.style={before row=\toprule,after row=\midrule},
    every last row/.style={after row=\bottomrule},
    ]{\Coatis}
  \label{tab:csvEstimacionCoati}
\end{table}

\begin{figure}[H]
  \centering
  \includegraphics[width=\textwidth]{../data/plot_pred_grid_multisession_Coati.png}
  \bicaption{Distribución de la población de coati.}{Distribution of the coati population.}
  \label{fig:coati_population_distribution}
\end{figure}

\begin{figure}[H]
  \centering
  \includegraphics[width=\textwidth]{../data/plot_cameras_for_month.png}
  \bicaption{Celdas con datos de cámaras trampa para el mes de \VAR{fecha_prediccion}.}{Grid cells with
  camera-trap data for the month of \VAR{prediction_date}.}
  \label{fig:cameras_per_month}
\end{figure}

\begin{table}[H]
  \centering
  \bicaption{Número de sesiones con datos de \VAR{fecha_inicio} a \VAR{fecha_fin} para cada celda.}{
    Number of sessions with data from \VAR{start_date} to \VAR{end_date} per grid cell.
  }
   \pgfplotstabletypeset[
     string type,
     columns={Grid,number_session},
     columns/{Grid}/.style={column name={\textbf{Celdas}}},
     columns/{number_session}/.style={column name={\textbf{Número de sesiones}}},
    every head row/.style={before row=\toprule,after row=\midrule},
    every last row/.style={after row=\bottomrule},
    ]{\CamerasSession}
  \label{tab:csvCamerasSessiones}
\end{table}

\bibliography{ref}
\bibliographystyle{apalike}
\end{document}
